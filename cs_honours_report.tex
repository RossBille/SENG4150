%
% Honours Report Template
% updated May 2013
%
\documentclass[a4,12pt]{article}
%
\usepackage{epsfig}
\usepackage{latexsym}
\usepackage{graphicx}
\usepackage{./natbib/natbib}
% It also sets the bibliographystyle to plainnat; for more information on
% natbib citation styles, see the natbib documentation, a copy of which
% is archived at http://www.jmlr.org/format/natbib.pdf
\usepackage{setspace}
\usepackage{amsmath}
\usepackage{amssymb}
\usepackage{amsfonts}
\usepackage{color}

%\graphicspath{{./figures/}
%Formatting-------------------------------------------------------------------
%\renewcommand{\refname}{\textbf{Literature}}
%
\renewcommand{\contentsname}{\small\textbf{{\center Table of Contents}}}
%
\setlength{\textheight}{8.8in}
%
\setlength{\topmargin}{-1.5cm}
%
\doublespacing
%\setlength{\textwidth}{17cm}
%
%\setlength{\oddsidemargin}{-0.1714in}
%
% Boxit -----------------------------------------------------------
\setlength{\fboxrule}{0.2mm} \setlength{\fboxsep}{4mm}
%
\newsavebox{\savepar}
\newenvironment{boxit}{\begin{lrbox}{\savepar}
        \begin{minipage}[b]{4.6in}}
        {\end{minipage}\end{lrbox}\fbox{\usebox{\savepar}}}
        
        
        
        
 \hyphenation{op-tical net-works Mathe-ma-tical street-scape street-scapes aes-the-tics aes-the-tic com-pu-ting geo-metric Geo-me-tric geo-metry boun-da-ries de-ve-lop-ment know-ledge mani-fold mani-folds high-di-men-sio-nal}
%
%
%
% Document-----------------------------------------------------------------------
%
\begin{document}
%
\title{\bf Distributed Operating Systems - Persistence \& State Preserveration}
%
\author{Ross Bille - 3127333\\
School of Electrical Engineering \& Computer Science\\
The University of Newcastle\\ Callaghan NSW 2308, Australia\\
Email: \texttt{c3127333@uon.edu.au} } 

\maketitle


\newpage
\begin{abstract}%
\noindent The abstract summarises the content of the paper or report and should have 70-200 words (depending on the publisher or other requirements); It
should state briefly what the paper is about (maybe also what
methods were used), what its (new) results are, why it is
important or significant. It can also be useful to state (or
indicate implicitly) who is the addressed readership and whether
its a review article, a short paper, a pilot study, an
extension of previous work or a thesis. Try to avoid special symbols, abbreviations, and citations.
\end{abstract}

\pagebreak

\tableofcontents

\pagebreak

\section{Introduction}

There are different ways to write an introduction. Typically it
contains background information and a review of literature which
indicates how the study fits into the context of other previous
work. This way the introduction can  address the significance and importance of the study. Major related publications in big journals should be cited
as well as closely related other articles. The literature review typically uses newer papers when it tries to address the state-of-the-art of a technique or recent developments. However, when first mentioning a method name the historically first source that introduced that concept should be cited. There are different citation styles and here is an example introduction
typically motivates the general hypotheses, aims and research questions
of a paper, report or thesis.  

\begin{center}
\begin{boxit}
\textbf{The research question is important.}
\end{boxit}
\end{center}

Questions raised in the introduction can later be answered in the final discussion.

The introduction often ends with a brief overview of the structure or organisation of
the paper, report or thesis.
test~\citep{ADearle}
%
\section{Persistence}\label{sec:persistence}
Generally in computer science persistence is when the data that a process is working on outlives the life of the aformentioned process.
\subsection{Orthogonal Persistence}
Also known as transparent persistence, orthogonal persistence is an implementation of persistence where the programmer doesn't need to invoke persistence in order to use it, for this to happen the persistence property should happen in the background, transparent to the programmer.
Three principles of persistence were identified by Atkinson and Morrison \citep{Atkinson}, these principles are as follows:
\begin{itemize}
    \item{The Principle of Persistence Independence - Programmer cannot invoke persistence}
    \item{The Principle of Data Type Orthogonality - All data objects are allowed to be persisted}
    \item{The Principle of Persistence Identification - The choice of how to identify persistent objects is orthogonal to the universe of discourse of the system}
\end{itemize}
\section{System State Preserveration}\label{sec:preservation}
\subsection{Low-power}
Keep a small amount of power in the system, just enough to keep primary memory live.
\subsection{Memory dump}
Write the entire system state to secondary storage.
\subsubsection{Hibernate}
User invokes an action to save the current system state to secondary storage.
as you will see in Section~\ref{sec:discussion}
\section{Discussion}\label{sec:discussion}
%compare and contrast
%
\section{Conclusion}
%
A brief final summary of the main achievements and outcomes. Possibly some suggestions for future work that can follow on from your project.%
%
\subsection*{Acknowledgements}
The author is grateful to A/Prof Frans Henskens and Mark Wallis.
%

\vskip 0.2in
%\newpage
\bibliographystyle{apalike}
\bibliography{./literature.bib}

\end{document}
